This chapter contains a mini-tutorial that has been used by Michael Jastram for the TdSE 2014 talk \href{http://www.tdse.org/programm2013/termine/icalrepeat.detail/2014/11/12/83/-/t2-modellgetriebene-systementwicklung-mit-eclipse.html}{Modellgetriebene Systementwicklung mit Eclipse}.

% ==================================================================
\section{Overview}
% ==================================================================

This tutorial covers the development of a small traffic light system, as shown in Figure~\ref{fig:trafficlight}.

\begin{figure}[h!]
  \centering
  \includegraphics[width=\linewidth]{../se-images/trafficlight.png}
  \caption{We will model a simple traffic light system.}
  \label{fig:trafficlight}
\end{figure}

% ==================================================================
\section{Tool Installation}
% ==================================================================

As of this writing, a complete toolchain is not yet available.  The following describes the installation from various components.

% ------------------------------------------------------------------
\subsection{Eclipse}
% ------------------------------------------------------------------

The basis for the toolchain are the \href{https://www.eclipse.org/downloads/packages/eclipse-modeling-tools/lunasr1}{Eclipse Modeling Tools}.  Please download for your platform and extract to a convenient location and start it.

\begin{info}
It may be not a bad idea to start with Polarsys instead, as it already includes Papyrus.
\end{info}

\begin{warning}On Linux, please edit eclipse.ini and add the following parameter \textbf{at the top (two lines}):

\begin{lstlisting}
--launcher.GTK_version
2
\end{lstlisting}
\end{warning}

% ------------------------------------------------------------------
\subsection{RMF and Formal Mind Essentials}
% ------------------------------------------------------------------

Next install the RMF (requirements) tools, but the repackaged version from Formal Mind:

\begin{itemize}
\item Use this update site: http://update.formalmind.com/studio
\item Unselect ``Group items by category''
\item Select \textbf{only} ``Formal Mind Studio (Feature)''
\item Complete the installation.
\end{itemize}

\begin{info}
The software is currently not signed, which will generate a warning.  Please continue with the installation, in spite of this.
\end{info}

% ------------------------------------------------------------------
\subsection{Java FX}
% ------------------------------------------------------------------

If you want to use rich text in requirements, you need support for Java FX.  Follow these steps:

\begin{itemize}
\item Use this update site: http://download.eclipse.org/efxclipse/runtime-released/1.1.0/site
\item Optional: Unselect ``Group items by category''
\item Select \textbf{only} ``Runtime Bundle Collector Feature''
\item Complete the installation.
\end{itemize}

% ------------------------------------------------------------------
\subsection{RMF-EMF Traceability}
% ------------------------------------------------------------------

In the context of a public research project (itea openETCS), a traceability plug-in for connecting arbitrary EMF models has been developed.

\begin{itemize}
\item Use this update site: http://openetcs.ci.cloudbees.com/job/openETCS-tycho/lastSuccessfulBuild/artifact/tool/bundles/org.openetcs.releng.products/target/repository
\item Unselect ``Group items by category''
\item Select \textbf{only} ``Feature''
\item Complete the installation.
\end{itemize}

% ------------------------------------------------------------------
\subsection{Additional Modeling Components}
% ------------------------------------------------------------------

You can install additional components for modeling via \menu{Papyrus via Help | Install Modeling Components}.  For this tutorial, useful components include:

\begin{description}
\item[Ecore Tools.] Support diagram notation for Ecore models.
\item[Papyrus.] Supports UML and SysML.
\end{description}

% ------------------------------------------------------------------
\subsection{Team Support}
% ------------------------------------------------------------------

Eclipse supports a number of team environments.  We recommend the installation of the egit plugin, allowing to work with git repositories.  The installation is described in the \href{http://formalmind.com/handbook?page=sec-versioning.html}{formalmind Studio Handbook}.

% ------------------------------------------------------------------
\subsection{Tool Configuration}
% ------------------------------------------------------------------

We recommend to switch to the ProR perspective, to get started.

% ==================================================================
\section{Import Requirements}
% ==================================================================

Typically, you already have requirements available in some form.  ProR includes a simple CSV-Importer that allows you to import existing requirements.  Follow these steps:

\begin{itemize}
\item Create a new Project via \menu{File | New | Project... | General | Project}
\item Call it \menu{tdse-1}
\item Create a new Requirements Model via right-click on the project, selecting \menu{New | Reqif10 Model}
\item Call the Model \menu{Trafficlight.reqif}
\item Import the .csv file via \menu{File | Import | formalmind Studio | CSV}
\item Create a mapping for the two columns to String attributes, as shown in Figure~\ref{fig:csv-import}
\end{itemize}

\begin{figure}[h!]
  \centering
  \includegraphics[width=\linewidth]{../se-images/csv-import.png}
  \caption{Result: The requirement is now a sibling of the chosen requirement.}
  \label{fig:csv-import}
\end{figure}

After the import, the new requirements have been added to the existing requirements specification.  There are a number of recommended improvements, for instance:

\begin{itemize}
\item Add a SpecObjectType for Headlines and configure the Headline Presentation, so that you can structure the text
\item Once you create headlines, you can arrange requirements as child elements (instead of siblings) under them.
\item You can create an information SpecObjectType, using XHTML and no IDs.  Use one of these to insert Figure~\ref{fig:trafficlight} into your specification (\href{../se-materials/tutorial/trafficlight.png}{trafficlight.png}).
\item Configure the ID Presentation to automatically create IDs for requirements, and center-align the ID.
\item Use the ID as a label (if available) by adjusting the Label Configuration.
\end{itemize}

The resulting specification is shown in Figure~\ref{fig:tutorial-step01}.

\begin{figure}[h!]
  \centering
  \includegraphics[width=\linewidth]{../se-images/tutorial-step01.png}
  \caption{The spec after completion of all steps so far.}
  \label{fig:tutorial-step01}
\end{figure}

% ==================================================================
\section{Glossary}
\label{sec:tutorial-glossary}
% ==================================================================

A glossary helps keeping track of terminology.  In this section, the glossary management from formalmind Studio is introduced, which supports color highlighting in the requirements text.

Note that this kind of glossary is a ``dead end'', in the sense that it cannot be used beyond its purpose.  Contrast that with a model-based data dictionary, as described in Section~\ref{sec:tutorial-ecore}.

The glossary is kind of cumbersome to configure.   Therefore, we included a correctly configured \href{../se-materials/tutorial/tdse-2}{Sample Project}.  Note that you need both the Highlighting and Keyword Highlighting presentations, in that order.

Figure~\ref{fig:tutorial-step02} shows the glossary, and its application to the requirements, which have been rewritten to use the terminology of the specification.

\begin{figure}[h!]
  \centering
  \includegraphics[width=\linewidth]{../se-images/tutorial-step02.png}
  \caption{Glossary Management in action.}
  \label{fig:tutorial-step02}
\end{figure}

In the screenshot, REQ-3 is being edited.  This results in the word \textit{green} being underlined in red, indicating that it is a recognized glossary entry.  The syntax highlighting diapears when not in edit mode.  Square brackets make a glossary term explicit.  If a term is marked that way that is not in the glossary, then it is shown in red.

% ==================================================================
\section{Data Dictionary with Ecore}
\label{sec:tutorial-ecore}
% ==================================================================

Ecore is the modeling language of the Eclipse Modeling Framework.  It has some similarities to UML Class diagrams, and is therefore well-suited for creating a precise data model.  It has the following advantages:

\begin{description}
\item[Easy to learn.] Especially if you already know class diagrams, you should be able to quickly learn Ecore.
\item[Code generation.] EMF allow the generation of Java code from Ecore models.  You can even generate a GUI based on a tree view.
\item[Test stub generation.] EMF allows the generation of test code stubs, making it easy to cover the unit test level.
\end{description}

On the other hand, it has its limitations.  In particular, it is not really possible to model dynamic aspects of the system.

\begin{warning}
While it is possible to mix this approach with the glossary management described in Section~\ref{sec:tutorial-glossary}, we do not recommend it, as it would lead to redundancy.  Redundancies should be avoided (DRY-principle: Don't Repeat Yourself).
\end{warning}

% ------------------------------------------------------------------
\subsection{Creating the Ecore Model}
% ------------------------------------------------------------------

We recommend to create a new Ecore Modeling Project via \menu{File | New | Project... | Eclipse Modeling Framework | Ecore Modeling Project}.   This way, everything will be properly configured for code generation and other cool stuff.  The model we use is shown in the right pane of Figure~\ref{fig:tutorial-step03}.

\begin{figure}[h!]
  \centering
  \includegraphics[width=\linewidth]{../se-images/tutorial-step03.png}
  \caption{On the left the requirements with links into the Ecore-based data model, shown on the right.}
  \label{fig:tutorial-step03}
\end{figure}

As you can see, the editors are arranged so that the requirements editor and the Ecore editor are visible at the same time.  This is necessary, as links are created by dragging model elements from the Ecore model onto the requirements.

\begin{info}
Linking via Drag and Drop is a feature taken from the openETCS project, \href{https://github.com/openETCS/toolchain/wiki/User-Documentation#Tracing_Requirements_and_SysML_Models}{where it is documented}.
\end{info}

We provided a prconfigured \href{../se-materials/tutorial/tdse-3}{Sample Project}, that allows annotating traces, as also shown in Figure~\ref{fig:tutorial-step03} (the second link of REQ-2).


% ==================================================================
\section{Modeling with Papyrus}
% ==================================================================


